%---------------------------------------------------------------------------------------------------
\documentclass[a4paper, 10pt, twoside]{article}
\input{../../Lectures/latex/style_art_es_1.tex}

%---------------------------------------------------------------------------------------------------
% \SetWatermarkText{\Sexpr{REP_watermark}}
% \SetWatermarkScale{0.35}
\SetWatermarkText{}
\SetWatermarkColor[cmyk]{0, 0, 0, 0.15}

% Bibliography -------------------------------------------------------------------------------------
\addbibresource{../../Lectures/bibtex/bibliography_articles.bib}
\addbibresource{../../Lectures/bibtex/bibliography_books.bib}

% Title --------------------------------------------------------------------------------------------
\title{\fontsize{20}{20}\selectfont \textbf{Modelación de epidemias}}
\author{Pedro Guarderas}
\date{\today}
\makeindex
\sloppy

%---------------------------------------------------------------------------------------------------
\begin{document}
\pagestyle{artsecstyle}

\pagenumbering{arabic}
\maketitle
\tableofcontents

\begin{abstract}
Not yet prepared \ldots
\end{abstract}

%---------------------------------------------------------------------------------------------------
\clearpage
\pagestyle{artstyle}
\section{Introducción}

\begin{definition}
Una \emph{epidemia} es una enfermedad que ataca a un gran número de seres vivos en mismo lugar y 
durante un mismo período de tiempo.
\end{definition}

%---------------------------------------------------------------------------------------------------
\clearpage
\section{Modelos basados en ecuaciones diferenciales}
Para interés de la epidemiología, no basta con encontrar y presentar una solución numérica al 
problema propuesto, mucho más importante es estudiar su dinámica y comportamiento, la solución 
numérica, no responde ciertas preguntas de carácter cualitativo que caracterizan al sistema 
dinámico.

\subsection{Consideraciones previas}
\begin{enumerate}
	\item $S=$ Susceptibles
	\item $E=$ Expuestos
	\item $I=$ Infectados
	\item $R=$ Resistentes
	\item $M=$ Portadores
\end{enumerate}

El teorema de Hartmann-Grobman justifica el estudio de sistemas no lineales mediante ciertas 
aprocimaciones lineales.

%---------------------------------------------------------------------------------------------------
\subsection{Modelos dinámicos clásicos}
Caracterizar el comportamiento de la enfermedad que produce la epidemia. 

Tasa de recuperación, tasa de mortalidad de enfermes, tasa de infección, nivel de incubación, 
distribución espacial, características patológicas, duración.

Es importante responder si la epidemia se va a propagar

%---------------------------------------------------------------------------------------------------
\subsubsection{Modelo SIR}
tasa de infección $\alpha > 0$ y la tasa de recuperación $\beta > 0$.
\begin{align}
\frac{dS}{dt} & = -\alpha S I \\
\frac{dI}{dt} & = \alpha S I - \beta I \nonumber \\
\frac{dR}{dt} & = \beta I \nonumber
\end{align}

%---------------------------------------------------------------------------------------------------
\subsection{Estudio cualitativo}

Determinación de puntos de equilibrio

Total de población enferma

%---------------------------------------------------------------------------------------------------
\subsection{Modelos dinámicos que consideran la difusión espacial}
En general se puede establecer un modelo bastante más elaborado que tome en cuenta la dispersión 
espacial de una epidemia.
\begin{align}
\frac{\partial S}{\partial t} & = d_1 \Delta S + f_1( S, E, I, R) \\
\frac{\partial E}{\partial t} & = d_2 \Delta E + f_2( S, E, I, R) \nonumber \\
\frac{\partial I}{\partial t} & = d_3 \Delta I + f_3( S, E, I, R) \nonumber \\
\frac{\partial R}{\partial t} & = d_4 \Delta R + f_4( S, E, I, R) \nonumber
\end{align}

$\Delta S = \dfrac{\partial^2 S}{\partial x^2} + \dfrac{\partial^2 S}{\partial y^2}$

Condiciones de borde Neumann homegéneas.
\begin{equation}
\frac{\partial S}{\partial \eta}( x, t ) = 
\frac{\partial E}{\partial \eta}( x, t ) = 
\frac{\partial I}{\partial \eta}( x, t ) = 
\frac{\partial R}{\partial \eta}( x, t ) = 0 \qquad \forall x \in \partial \Omega
\end{equation}

Condiciones iniciales.
\begin{equation}
S( x, 0 ) = S_0( x ),\quad E( x, 0 ) = E_0( x ),\quad I( x, 0 ) = I_0,\quad R( x, 0 ) = R_0( x )
\end{equation}

%---------------------------------------------------------------------------------------------------
% Bibliography
\clearpage
\textcite{accu_stab_num_alg, fun_ergo_theo, pde_evans, trans_equ_bio_2007, 
	real_comp_ana_1987, func_ana_1987}

\section{Bibliografía}
\printbibliography[heading = none]
 
\end{document}
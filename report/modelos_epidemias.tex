%---------------------------------------------------------------------------------------------------
\documentclass[a4paper, 10pt, twoside]{article}
\input{../../../Lectures/latex/style_art_es_1.tex}

%---------------------------------------------------------------------------------------------------
% \SetWatermarkText{\Sexpr{REP_watermark}}
% \SetWatermarkScale{0.35}
\SetWatermarkText{}
\SetWatermarkColor[cmyk]{0, 0, 0, 0.15}

% Bibliography -------------------------------------------------------------------------------------
\addbibresource{../../Lectures/bibtex/bibliography_articles.bib}
\addbibresource{../../Lectures/bibtex/bibliography_books.bib}

% Title --------------------------------------------------------------------------------------------
\title{\fontsize{20}{20}\selectfont \textbf{Modelación de epidemias}}
\author{Pedro Guarderas}
\date{\today}
\makeindex
\sloppy

%---------------------------------------------------------------------------------------------------
\begin{document}
\pagestyle{artsecstyle}

\pagenumbering{arabic}
\maketitle
\tableofcontents

\begin{abstract}
Not yet prepared \ldots
\end{abstract}

%---------------------------------------------------------------------------------------------------
\clearpage
\pagestyle{artstyle}
\section{Introducción}

Una \emph{epidemia} es una enfermedad que ataca a un gran número de personas o de animales en un 
mismo lugar durante un período de tiempo. Cuando la enfermedad supera las barreras de varios
países se habla de un \emph{pandemia}. 
\newline

La \emph{epidemiología teórica} se encarga de generar modelos matemáticos para representar la 
enfermedad.
\newline

Para interés de la epidemiología, no basta con encontrar y presentar una solución numérica al 
problema propuesto, mucho más importante es estudiar su dinámica y comportamiento, la solución 
numérica, no responde ciertas preguntas de tipo cualitativo que caracterizan al sistema 
dinámico y por tanto la evolución de la enfermedad.

La creación de un modelo matemático exige tener en cuenta los siguientes pasos.
\begin{enumerate}
	\item Conceptualización
	\item Análisis
	\item Estimación
	\item Cálculo
	\item Calibración
\end{enumerate}

%---------------------------------------------------------------------------------------------------
\section{Modelos continuos}
Para interés de la epidemiología, no basta con encontrar y presentar una solución numérica al 
problema propuesto, mucho más importante es estudiar su dinámica y comportamiento, la solución 
numérica, no responde ciertas preguntas de carácter cualitativo que caracterizan al sistema 
dinámico.

\begin{enumerate}
	\item $S=$ Susceptibles
	\item $E=$ Expuestos
	\item $I=$ Infectados
	\item $R=$ Resistentes
	\item $M=$ Portadores
\end{enumerate}

El teorema de Hartmann-Grobman justifica el estudio de sistemas no lineales mediante ciertas 
aprocimaciones lineales.

%---------------------------------------------------------------------------------------------------
\subsection{Modelos dinámicos clásicos}
Caracterizar el comportamiento de la enfermedad que produce la epidemia. 

Tasa de recuperación, tasa de mortalidad de enfermes, tasa de infección, nivel de incubación, 
distribución espacial, características patológicas, duración.

Es importante responder si la epidemia se va a propagar

%---------------------------------------------------------------------------------------------------
\subsection{Modelo SIR}
Modelo propuesto por William Kermack y Anderson McKendrick alrededor de 1927. Es uno de los primeros
modelos continuos para epidemias, establece un sistema dinámico para la evolución de una población 
donde los individuos son caracterizados como: susceptibles $S$, infectados $I$ y resistentes $R$.
\begin{align}
\frac{dS}{dt} & = -\alpha S I \nonumber \\
\frac{dI}{dt} & = \alpha S I - \beta I \\
\frac{dR}{dt} & = \beta I \nonumber
\end{align}
Donde, $\alpha > 0$ es la \emph{tasa de infección} y $\beta > 0$ es la \emph{tasa de recuperación}.
\newline
El modelo diferencial para estar completamente determinado se requiere estimar los parámetros 
$\alpha$, $\beta$, establecer condiciones iniciales $S(0) = S_0, I(0) = I_0, R(0) = R_0$ y 
un horizonte de proyección prudencial $T$ > 0.

%---------------------------------------------------------------------------------------------------
\subsection{Modelo SIR con mortalidad}
Cuando la enfermedad puede llevar a la muerte, es importante incluir este factor en el modelo; 
tomando en cuenta la tasa de mortalidad propia de la población $\mu$ y la mortalidad debida
a la enfermedad $\eta$.
\begin{align}
\frac{dS}{dt} & = -\alpha S I + \mu ( 1 - S ) + \eta \beta I \nonumber \\
\frac{dI}{dt} & = \alpha S I - ( \mu + \beta ) I \\
\frac{dR}{dt} & = ( 1 - \eta ) \beta I - \mu R \nonumber
\end{align}

%---------------------------------------------------------------------------------------------------
\subsection{Número básico de reproducción}
Número esperado de casos secundarios producidos por un individuo infectado durante su periodo total
de infección, en una población solo compuesta por individuos susceptibles. La fórmula tiene origen 
en los modelos de natalidad utilizados en demografía.
\begin{equation}
\mathcal{R}_0 = \int\limits_0^\infty b( t ) F( t )\, dt
\end{equation}

Donde $F$ es la probabilidad de continuar infectado si se está infectado y $b$ es el número medio 
de nuevos infectados producidos por un individuo infectado.

Para el caso del modelo SIR, $F( t ) = e^{-\beta t}$ y $b( t ) = \alpha$.
\begin{equation}
\mathcal{R}_0 = \int\limits_0^\infty \alpha e^{-\beta t}\, dt = \frac{\alpha}{\beta}
\end{equation}

Para el caso del modelo SIR con mortalidad, $F( t ) = e^{-(\beta + \mu ) t}$ y $b( t ) = \alpha$.
\begin{equation}
\mathcal{R}_0 = \int\limits_0^\infty \alpha e^{-(\beta + \mu) t}\, dt =
\frac{\alpha}{\beta + \mu}
\end{equation}

%---------------------------------------------------------------------------------------------------
\subsection{Estudio cualitativo}

Determinación de puntos de equilibrio

Total de población enferma

%---------------------------------------------------------------------------------------------------
\subsection{Modelos dinámicos que consideran la difusión espacial}
En general se puede establecer un modelo bastante más elaborado que tome en cuenta la dispersión 
espacial de una epidemia.
\begin{align}
\frac{\partial S}{\partial t} & = d_1 \Delta S + f_1( S, E, I, R) \\
\frac{\partial E}{\partial t} & = d_2 \Delta E + f_2( S, E, I, R) \nonumber \\
\frac{\partial I}{\partial t} & = d_3 \Delta I + f_3( S, E, I, R) \nonumber \\
\frac{\partial R}{\partial t} & = d_4 \Delta R + f_4( S, E, I, R) \nonumber
\end{align}

$\Delta S = \dfrac{\partial^2 S}{\partial x^2} + \dfrac{\partial^2 S}{\partial y^2}$

Condiciones de borde Neumann homegéneas.
\begin{equation}
\frac{\partial S}{\partial \eta}( x, t ) = 
\frac{\partial E}{\partial \eta}( x, t ) = 
\frac{\partial I}{\partial \eta}( x, t ) = 
\frac{\partial R}{\partial \eta}( x, t ) = 0 \qquad \forall x \in \partial \Omega
\end{equation}

Condiciones iniciales.
\begin{equation}
S( x, 0 ) = S_0( x ),\quad E( x, 0 ) = E_0( x ),\quad I( x, 0 ) = I_0,\quad R( x, 0 ) = R_0( x )
\end{equation}

\section{Observaciones}
Muchos resultados de sistemas dinámicos son necesarios para comprender el comportamiento de los 
modelos epidemiológicos. Así se puede comprender la estabilidad, equilibrio, incidencia máxima.

Debido a la no linealidad de estos modelos, para realizar los cálculos numéricos se requiere
de algoritmos debidamente implementados, con alta precisión y estabilidad.

El enfrentar una epidemia requiere manejo honesto, disciplinado y diligente de la información 
estadística, esta será muy útil para la actualización y generación de modelos y así tomar 
decisiones de salud pública adecuadas y oportunas.

Las epidemias tienden a desarrollarse más rápido si la densidad de susceptibles es alta, por 
ejemplo, sobrepoblación, y si la tasa de retiro, $\beta$, es baja, por ejemplo, tratamiento médico 
insuficiente.

%---------------------------------------------------------------------------------------------------
% Bibliography
\clearpage
\textcite{accu_stab_num_alg, fun_ergo_theo, pde_evans, trans_equ_bio_2007, 
	real_comp_ana_1987, func_ana_1987}

\section{Bibliografía}
\printbibliography[heading = none]
 
\end{document}